
\subsection*{Objective functional}

The objective is the most important term in the performance functional expression. It defines the task that the neural
network is required to accomplish.

For data modeling applications, such as function regression or pattern recognition, the sum squared error is the reference objective functional. It measures the diffence between the outputs from a neural network and the targets in a data set.  
Some related objective functionals here are the normalized squared error or the Minkowski error. 

Applications in which the neural network learns from a mathematical model require other objective functionals. 
For instance, we can talk about minimum final time or desired trajectory optimal control problems. 
That two performance terms are called in \texttt{OpenNN} independent parameters error and solutions error, respectively.

\subsection*{Regularization functional}

A problem is called well-possed if its solution meets
existence, uniqueness and stability. A solution is said to be stable when
small changes in the independent variable led to
small changes in the dependent variable.
Otherwise the problem is said to be ill-possed. 

An approach for ill-possed problems is to control the effective complexity of the neural network
\cite{Tikhonov1977}. This can be achieved by using a regularization term into the performance functional.  

One of the simplest forms of regularization term consists on the norm of the neural parameters vector \cite{Bishop1995}. 
Adding that term to the objective functional will cause the neural network to have
smaller weights and biases, and this will force its response to be
smoother.

Regularization can be used in problems which learn from a data set. 
Function regression or pattern recognition problems with noisy data sets are common applications.  
It is also useful in optimal control problems which aim to save control action. 
More information of regularization theory for neural networks can be found in \cite{Girosi1995} and \cite{Chen2002}.

\subsection*{Constraints functional}

A variational problem for a neural network can be
specified by a set of constraints, which are equalities or
inequalities that the solution must
satisfy. Such constraints are expressed as
functionals. 

Here the aim is to find a solution which makes all the constraints to be satisfied and the objective functional to be an extremum.

Constraints are required in many optimal control or optimal shape design problems. 
For instance, we can talk about lenght, area or volume constraints.
That type of performance term is called in \texttt{OpenNN} final solutions error.

\subsection*{Performance functional}

The performance measure is a functional of the neural network which can take the following forms:

\begin{eqnarray}\nonumber
\text{performance functional} &=& \text{Functional[neural network]},\\\nonumber
\text{performance functional} &=& \text{Functional[neural network, data set]},\\\nonumber
\text{performance functional} &=& \text{Functional[neural network, mathematical model]},\\\nonumber 
\text{performance functional} &=& \text{Functional[neural network, mathematical model, data set]}.
\end{eqnarray}


In order to perform a particular task a neural network must
be associated a performance functional, which depends on the variational
problem at hand. The learning problem 
is thus formulated in terms of the minimization of the performance
functional.

The performance functional defines the task that the neural
network is required to accomplish and provides a measure of the
quality of the representation that the neural network is required to
learn. In this way, the choice of a suitable performance functional
depends on the particular application.

The learning problem can then be stated as to find a neural network for which the performance functional 
takes on a minimum or a maximum value. This is a variational problem. 

In the context of neural network, the variational problems are 
can be treated as a function optimization problem. 
The variational approach looks at the performance as being a functional of the function represented by the neural network.
The optimization approach looks at the performance as being a function of the parameters in the neural network.  

Then, a performance function can be visualized as a hypersurface,
with the neural network parameters as coordinates, see Figure
\ref{PerformanceFunctionFigure}.

The performance function evaluates the performance of the neural network by looking at its parameters. 
More complex calculations allow to obtain some partial derivatives of the performance with respect to the parameters.
The first partial derivatives are arranged in the gradient vector. The second partial derivatives are arranged in the Hessian matrix. 

When the desired output of the neural network for a given
input is known, the gradient and Hessian can usually be found
analytically using back-propagation. In some other circumstances
exact evaluation of that quantities is not possible and numerical
differentiation must be used.

\begin{eqnarray}\nonumber
\text{performance functional} =  \text{objective term} + \text{regularization term} + \text{constraints term}
\end{eqnarray}

