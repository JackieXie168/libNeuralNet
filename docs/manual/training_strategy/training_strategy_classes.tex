\texttt{OpenNN} includes the class \lstinline"TrainingStrategy" to represent the concept of training strategy. 

\subsection*{Members}

The training strategy class contains:

\begin{itemize}
\item[-] A pointer to a performance functional object.
\item[-] A pointer to an initialization training algorithm.
\item[-] A pointer to a main training algorithm.
\item[-] A pointer to a refinement training algorithm.
\item[-] The type of initialization training algorithm.
\item[-] The type of main training algorithm.
\item[-] The type of refinement training algorithm.
\item[-] A flag for using the initialization training algorithm.
\item[-] A flag for using the main training algorithm.
\item[-] A flag for using the refinement training algorithm.
\end{itemize}

All members are private, and must be accessed or modified by means of get and set methods, respectively. 

\subsection*{Constructors}

To construct a training strategy object associated to a performance functional object we do the following:

\begin{lstlisting}
TrainingStrategy ts(&pf);
\end{lstlisting}

\noindent where \lstinline"pf" is some performance functional object. 

\subsection*{Methods}

The default training strategy consists on a main training algorithm of the quasi-Newton method type. 
The next sentence sets a different training strategy.

\begin{lstlisting}
RandomSearch* rsp = new RandomSearch(&pf);
rsp->set_epochs_number(10);
ts.set_initialization_training_algorithm(rsp);


GradientDescent* gdp = new GradientDescent(&pf);
gdp->set_epochs_number(25);
ts.set_main_training_algorithm(gdp);
\end{lstlisting}


The most important method of a training strategy is called \lstinline"perform_training". 
The use is as follows:

\begin{lstlisting}
ts.perform_training();
\end{lstlisting}

\noindent where \lstinline"ts" is some training strategy object. 

We can save the above object to a XML file. 

\begin{lstlisting}
ts.save("training_strategy.xml");
\end{lstlisting}


\subsection*{XML formats}

In this section we list the XML formats of the training strategy classes in \texttt{OpenNN}. 

\subsubsection*{Training rate algorithm}

Some training algorithms contain a training rate object inside. 
The format of this object is listed below. 

\begin{lstlisting}
<TrainingRateAlgorithm>
   <TrainingRateMethod>string</TrainingRateMethod>
   <BracketingFactor>real</BracketingFactor>
   <FirstTrainingRate>real</FirstTrainingRate>
   <Display>boolean</Display>
</TrainingRateAlgorithm> 
\end{lstlisting}

\subsubsection*{Gradient descent}
   
The file format of this class is listed below. 

\begin{lstlisting}
<GradientDescent>
   <TrainingRate>
   training_rate_element
   </TrainingRate>
   <WarningTrainingRate>real</WarningTrainingRate>
   <ErrorTrainingRate>real<ErrorTrainingRate>
   
   <MinimumParametersIncrementNorm>real</MinimumParametersIncrementNorm>
   <EvaluationGoal>real</EvaluationGoal>
   <MinimumEvaluationImprovement>real</MinimumEvaluationImprovement>
   <GradientNormGoal>real</GradientNormGoal>
   <MaximumEpochsNumber>integer</MaximumEpochsNumber>
   <MaximumTime>real</MaximumTime>
   
   <ReserveElapsedTimeHistory>boolean</ReserveElapsedTimeHistory>
   <ReserveParametersHistory>boolean</ReserveParametersHistory>
   <ReserveParametersNormHistory>boolean</ReserveParametersNormHistory>
   <ReservePerformanceHistory>boolean</ReserveEvaluationHistory>
   <ReserveValidationErrorHistory>boolean</ReserveValidationErrorHistory>
   <ReserveGradientHistory>boolean</ReserveGradientHistory>
   <ReserveGradientNormHistory>boolean</ReserveGradientNormHistory>
   <ReserveTrainingDirectionHistory>boolean</ReserveTrainingDirectionHistory>
   <ReserveTrainingDirectionNormHistory>boolean</ReserveTrainingDirectionNormHistory>
   <ReserveTrainingRateHistory>boolean</ReserveTrainingRateHistory>
   
   <WarningParametersNorm>real</WarningParametersNorm>
   <WarningGradientNorm>real</WarningGradientNorm>
   <Display>boolean</Display>
   <DisplayPeriod>integer</DisplayPeriod>
</GradientDescent>   
\end{lstlisting}

\subsubsection*{Newton method}

The file format of this class is listed below. 

\begin{lstlisting}
<NewtonMethod>
   <TrainingRateMethod>
   training_rate_method
   </TrainingRateMethod>
         
   <WarningParametersNorm>real</WarningParametersNorm>
   <WarningGradientNorm>real</WarningGradientNorm>
   <WarningTrainingRate>real</WarningTrainingRate>
   
   <ErrorParametersNorm>real</ErrorParametersNorm>
   <ErrorGradientNorm>real</ErrorGradientNorm>
   <ErrorTrainingRate>real</ErrorTrainingRate>
   
   <MinimumParametersIncrementNorm>real</MinimumParametersIncrementNorm>
 
   <MinimumEvaluationImprovement>real</MinimumEvaluationImprovement>
   <EvaluationGoal>real</EvaluationGoal>
 
   <GradientNormGoal>real</GradientNormGoal>
   
   <MaximumEpochsNumber>integer</MaximumEpochsNumber>
   
   <MaximumTime>real</MaximumTime>
   
   <ReserveParametersHistory>boolean</ReserveParametersHistory>   
   <ReserveParametersNormHistory>boolean</ReserveParametersNormHistory>

   <ReserveEvaluationHistory>boolean</ReserveEvaluationHistory>   
   <ReserveGeneralizationErrorHistory>boolean</ReserveGeneralizationErrorHistory>
   <ReserveGradientHistory>boolean</ReserveGradientHistory>
   <ReserveGradientNormHistory>boolean</ReserveGradientNormHistory>
   <ReserveInverseHessianHistory>boolean</ReserveInverseHessianHistory>
   
   <ReserveTrainingDirectionHistory>boolean</ReserveTrainingDirectionHistory>
   <ReserveTrainingDirectionNormHistory>boolean</ReserveTrainingDirectionNormHistory>
   <ReserveTrainingRateHistory>boolean</ReserveTrainingRateHistory>
   <ReserveElapsedTimeHistory>boolean</ReserveElapsedTimeHistory>
   
   <Display>boolean</Display>
   <DisplayPeriod>integer</DisplayPeriod>
</NewtonMethod>
\end{lstlisting}


\subsubsection*{Conjugate gradient}

The file format of the conjugate gradien object is as follows:

\begin{lstlisting}
<ConjugateGradient>
   <TrainingDirectionMethod>string</TrainingDirectionMethod>
   
   <TrainingRateAlgorithm>
   training_rate_algorithm_element
   </TrainingRateAlgorithm>
      
   <WarningParametersNorm>real</WarningParametersNorm>
   <WarningGradientNorm>real</WarningGradientNorm>
   <WarningTrainingRate>real</WarningTrainingRate>
   
   <ErrorParametersNorm>real</ErrorParametersNorm>
   <ErrorGradientNorm>real</ErrorGradientNorm>
   <ErrorTrainingRate>real</ErrorTrainingRate>
   
   <MinimumParametersIncrementNorm>real</MinimumParametersIncrementNorm>
  
   <MinimumEvaluationImprovement>real</MinimumEvaluationImprovement> 
   <EvaluationGoal>real</EvaluationGoal>
   <GradientNormGoal>real</GradientNormGoal>
   
   <MaximumEpochsNumber>integer</MaximumEpochsNumber>
   <MaximumTime>real</MaximumTime>
   
   <ReserveParametersHistory>boolean</ReserveParametersHistory>
   <ReserveParametersNormHistory>boolean</ReserveParametersNormHistory>
   
   <ReserveEvaluationHistory>boolean</ReserveEvaluationHistory>
   <ReserveGeneralizationEvaluationHistory>boolean</ReserveGeneralizationEvaluationHistory>
   <ReserveGradientHistory>boolean</ReserveGradientHistory>
   <ReserveGradientNormHistory>boolean</ReserveGradientNormHistory>
   
   <ReserveTrainingDirectionHistory>boolean</ReserveTrainingDirectionHistory>
   <ReserveTrainingDirectionNormHistory>boolean</ReserveTrainingDirectionNormHistory>
   <ReserveTrainingRateHistory>boolean</ReserveTrainingRateHistory>
   <ReserveElapsedTimeHistory>boolean</ReserveElapsedTimeHistory>
   
   <DisplayPeriod>integer</DisplayPeriod>
   <Display>boolean</Display>
</ConjugateGradient>
\end{lstlisting}


\subsubsection*{Quasi-Newton method XML format}

See below for the format of a quasi-Newton method XML-type file in \texttt{OpenNN}.

\begin{lstlisting}
<QuasiNewtonMethod version="3.0">

   <TrainingRateAlgorithm>
   training_rate_algorithm_element
   </TrainingRateAlgorithm>
   
   <WarningParametersNorm>real</WarningParametersNorm>
   <WarningGradientNorm>real</WarningGradientNorm>
   <WarningTrainingRate>real</WarningTrainingRate>
   
   <ErrorParametersNorm>real</ErrorParametersNorm>
   <ErrorGradientNorm>real</ErrorGradientNorm>   
   <ErrorTrainingRate>real</ErrorTrainingRate>
   
   <MinimumParametersIncrementNorm>real</MinimumParametersIncrementNorm>
   
   <MinimumEvaluationImprovement>real</MinimumEvaluationImprovement>
   <EvaluationGoal>real</EvaluationGoal>
   <GradientNormGoal>real</GradientNormGoal>
   
   <MaximumEpochsNumber>integer</MaximumEpochsNumber>
   <MaximumTime>real</MaximumTime>
   
   <ReserveParametersHistory>boolean</ReserveParametersHistory>
   <ReserveParametersNormHistory>boolean</ReserveParametersNormHistory>
   
   <ReserveEvaluationHistory>boolean</ReserveEvaluationHistory>
   <ReserveGeneralizationEvaluationHistory>boolean</ReserveGeneralizationEvaluationHistory>
   <ReserveGradientHistory>boolean</ReserveGradientHistory>   
   <ReserveGradientNormHistory>boolean</ReserveGradientNormHistory>
   <ReserveInverseHessianHistory>boolean</ReserveInverseHessianHistory>
   
   <ReserveTrainingDirectionHistory>boolean</ReserveTrainingDirectionHistory>
   <ReserveTrainingDirectionNormHistory>boolean</ReserveTrainingDirectionNormHistory>
   <ReserveTrainingRateHistory>boolean</ReserveTrainingRateHistory>
   <ReserveElapsedTimeHistory>boolean</ReserveElapsedTimeHistory>
   
   <Display>boolean</Display>
   
   <DisplayPeriod>integer</DisplayPeriod>
</QuasiNewtonMethod>
\end{lstlisting}

\subsubsection*{Levenberg Marquardt algorithm}

The file format of this class is listed below. 

\begin{lstlisting}
<LevenbergMarquardtAlgorithm>
   <LinearAlgebraicEquations>
   linear_algebraic_equations_element
   </LinearAlgebraicEquations>
   <DampingParameter>real</DampingParameter>
   <MinimumDampingParameter>real</MinimumDampingParameter>
   <MaximumDampingParameter>real</MaximumDampingParameter>
   <DampingParameterFactor>real</DampingParameterFactor>

   <WarningParametersNorm>real</WarningParametersNorm>
   <WarningGradientNorm>real</WarningGradientNorm>

   <ErrorParametersNorm>real</ErrorParametersNorm>
   <ErrorGradientNorm>real</ErrorGradientNorm>
 
   <MinimumParametersIncrementNorm>real</MinimumParametersIncrementNorm>
   <EvaluationGoal>real</EvaluationGoal>
   <MinimumEvaluationImprovement>real</MinimumEvaluationImprovement>
   <GradientNormGoal>real</GradientNormGoal>
   <MaximumEpochsNumber>integer</MaximumEpochsNumber>
   <MaximumTime>real</MaximumTime>
   
   <ReserveParametersHistory>boolean</ReserveParametersHistory>
   <ReserveParametersNormHistory>boolean</ReserveParametersNormHistory>

   <ReserveEvaluationHistory>boolean</ReserveEvaluationHistory>
   <ReserveGeneralizationEvaluationHistory>boolean</ReserveGeneralizationEvaluationHistory>
   <ReserveGradientHistory>boolean</ReserveGradientHistory>
   <ReserveGradientNormHistory>boolean</ReserveGradientNormHistory>
   
   <ReserveTrainingDirectionHistory>boolean</ReserveTrainingDirectionHistory>
   <ReserveTrainingDirectionNormHistory>boolean</ReserveTrainingDirectionNormHistory>
   <ReserveTrainingRateHistory>boolean</ReserveTrainingRateHistory>
   <ReserveElapsedTimeHistory>boolean</ReserveElapsedTimeHistory>
   
   <WarningGradientNorm>real</WarningGradientNorm>
   <Display>boolean</Display>
   <DisplayPeriod>integer</DisplayPeriod>
</LevenbergMarquardtAlgorithm>   
\end{lstlisting}

\subsubsection*{Random search}

The file format of this class is listed below. 

\begin{lstlisting}
<RandomSearch>

   <FirstTrainingRate>double</FirstTrainingRate>
   <TrainingRateReductionFactor>double</TrainingRateReductionFactor>
   <TrainingRateReductionPeriod>unsigned int</TrainingRateReductionPeriod>
   <WarningParametersNorm>double</WarningParametersNorm>
   <WarningTrainingRate>double</WarningTrainingRate>
   <ErrorParametersNorm>double</ErrorParametersNorm>
   <ErrorTrainingRate>double</ErrorTrainingRate>
   
   <MinimumParametersIncrementNorm>double</MinimumParametersIncrementNorm>
   <MinimumPerformanceIncrease>double</MinimumPerformanceIncrease>
   <PerformanceGoal>double</PerformanceGoal>
   <MaximumGeneralizationEvaluationDecreases>unsigned int</MaximumGeneralizationEvaluationDecreases>
   <MaximumEpochsNumber>unsigned int</MaximumEpochsNumber>
   <MaximumTime>double</MaximumTime>
   
   <ReserveParametersHistory>bool</ReserveParametersHistory>
   <ReserveParametersNormHistory>bool</ReserveParametersNormHistory>
   <ReserveEvaluationHistory>bool</ReserveEvaluationHistory>
   <ReserveGeneralizationEvaluationHistory>bool</ReserveGeneralizationEvaluationHistory>
   <ReserveTrainingDirectionHistory>bool</ReserveTrainingDirectionHistory>
   <ReserveTrainingDirectionNormHistory>bool</ReserveTrainingDirectionNormHistory>
   <ReserveTrainingRateHistory>bool</ReserveTrainingRateHistory>
   <ReserveElapsedTimeHistory>bool</ReserveElapsedTimeHistory>
   
   <DisplayPeriod>unsigned int</DisplayPeriod>
   <Display>bool</Display>
</RandomSearch>

\end{lstlisting}

\subsubsection*{Evolutionary algorithm}

The next listing shows the format of an evolutionary algorithm data file in \texttt{OpenNN}. 
It is of XML-type. 

\begin{lstlisting}
<EvolutionaryAlgorithm>

   Training operators

   <FitnessAssignmentMethod>string</FitnessAssignmentMethod>
   <SelectionMethod>string</SelectionMethod>
   <RecombinationMethod>string</RecombinationMethod>
   <MutationMethod>string</MutationMethod>
      
   Training parameters
   
   <PopulationSize>integer</PopulationSize>
   <Elitism>boolean</Elitism>
   <SelectivePressure>real</SelectivePressure>
   <RecombinationSize>real</RecombinationSize>
   <MutationRate>real</MutationRate>
   <MutationRange>real</MutationRange>
   
   Stopping criteria
   
   <EvaluationGoal>real</EvaluationGoal>
   <MeanEvaluationGoal>real</MeanEvaluationGoal>
   <StandardDeviationEvaluationGoal>real</StandardDeviationEvaluationGoal>
   
   <MaximumGenerationsNumber>real</MaximumGenerationsNumber>
   <MaximumTime>real</MaximumTime>
      
   Training history
   
   <ReservePopulationHistory>boolean</ReservePopulationHistory>
   <ReserveMeanNormHistory>boolean</ReserveMeanNormHistory>
   <ReserveStandardDeviationNormHistory>boolean</ReserveStandardDeviationNormHistory>
   <ReserveBestNormHistory>boolean</ReserveBestNormHistory>
   
   <ReserveMeanEvaluationHistory>boolean</ReserveMeanEvaluationHistory>
   <ReserveStandardDeviationEvaluationHistory>boolean</ReserveStandardDeviationEvaluationHistory>
   <ReserveBestEvaluationHistory>boolean</ReserveBestEvaluationHistory>
      
   <Display>boolean</Display>
   <DisplayPeriod>integer</DisplayPeriod>
   
</EvolutionaryAlgorithm>
\end{lstlisting}


\subsubsection*{Training strategy}

The XML format of a training strategy class is listed below.
It might contain up to three training algorithms. 

\begin{lstlisting}
<TrainingStrategy>

   <InitializationTrainingAlgorithm>
   initialization_training_algorithm_element
   </InitializationTrainingAlgorithm>
   <MainTrainingAlgorithm>
   main_training_algorithm_element
   </MainTrainingAlgorithm>
   <RefinementTrainingAlgorithm>
   refinement_training_algorithm_element
   </RefinementTrainingAlgorithm>

   <InitializationTrainingAlgorithmFlag>boolean</InitializationTrainingAlgorithmFlag>
   <MainTrainingAlgorithmFlag>boolean</MainTrainingAlgorithmFlag>
   <RefinementTrainingAlgorithmFlag>boolean</RefinementTrainingAlgorithmFlag>
       
   <Display>boolean</Display>
</TrainingStrategy>
\end{lstlisting}

