\index{Matrix class}

As it happens with the \lstinline"Vector" class, the \lstinline"Matrix" class is also a template \cite{Eckel2000}. 
Therefore, a \lstinline"Matrix" of any type can be created. 

\subsubsection*{Members}

The \lstinline"Matrix" class has three members:

\begin{itemize}
\item[-] The number of rows. 
\item[-] The number of columns. 
\item[-] A double pointer to some type. 
\end{itemize}

That members are private. Private members can be accessed only within methods of the class itself.

\subsubsection*{File format}

The member values of a matrix object can be serialized or deserialized to or from a data file. 
The format is as follows. 

\begin{lstlisting}
element_00 ... element_0M
...
element_N0 ... element_NM
\end{lstlisting}

\subsubsection*{Constructors}

The \lstinline"Matrix" class also implements multiple constructors, with different parameters.

% Default constructor

The default constructor creates a matrix with zero rows and zero columns, 

\begin{lstlisting}
Matrix<MyClass> m;
\end{lstlisting}

% Rows and columns numbers constructor

In order to construct an empty \lstinline"Matrix" with a specified number of rows and columns we use

\begin{lstlisting}
Matrix<int> m(2, 3);
\end{lstlisting}

% Initialization constructor

We can specify the number of rows and columns and initialize the \lstinline"Matrix" elements at the same time by doing

\begin{lstlisting}
Matrix<double> m(1, 5, 0.0);
\end{lstlisting}

% File constructor

To build a \lstinline"Matrix" object by loading its members from a data file the following constructor is used, 

\begin{lstlisting}
Matrix<double> m(`Matrix.dat');
\end{lstlisting}

The format of a matrix data file is as follows: the first line contains the numbers of rows and columns separated by a blank space; the following data contains the matrix elements arranged in rows and columns. For instance, the next data will correspond to a \lstinline"Matrix" of zeros with 2 rows and 3 columns,

\begin{lstlisting}
2 3
0 0 0
0 0 0
\end{lstlisting}

% Copy constructor

The copy constructor builds an object which is a copy of another object, 

\begin{lstlisting}
Matrix<bool> a(3,5);
Matrix<bool> b(a);
\end{lstlisting}

\subsubsection*{Operators}

% Assignment operator 

The \lstinline"Matrix" class also implements the assignment operator, 


\begin{lstlisting}
Matrix<double> a(2,1);
Matrix<bool> b = a;
\end{lstlisting}

% Reference operator 

Below there is an usage example of the reference operator here. Note that row indexing goes from \lstinline"0" to
\lstinline"rows_number-1"  and column indexing goes from \lstinline"0" to \lstinline"columns_number-1".

% Arithmetic operators 

\begin{lstlisting}
Matrix<int> m(2, 2);
m[0][0] = 1;
m[0][1] = 2;
m[1][0] = 3;
m[1][1] = 4;
\end{lstlisting}

% Arithmetic operators

The use of the arithmetic operators for the \lstinline"Matrix" class are very similar to those for the \lstinline"Vector" class. The following sentence uses the scalar difference operator, 

\begin{lstlisting}
Matrix<double> a(5, 7, 2.5);
Matrix<double> b = a + 0.1;
\end{lstlisting}

% Arithmetic and assignment operators

Also, using the arithmetic and assignment operators with the \lstinline"Matrix" class is similar than with the \lstinline"Vector" class. For instance, to assign by sum with another \lstinline"Matrix" we can write

\begin{lstlisting}
Matrix<double> a(1, 2, 1.0);
Matrix<double> b(1, 2, 0.5);
a += b;
\end{lstlisting}

% Equality and relational operators

The not equal to operator with another \lstinline"Matrix" can be used in the following way, 

\begin{lstlisting}
Matrix<std::string> a(1, 1, `hello');
Matrix<std::string> b(1, 1, `good bye');
bool is_not_equal_to = (a != b);
\end{lstlisting}

The use of the greater than operator with a scalar is listed below

\begin{lstlisting}
Matrix<double> a(2, 3, 0.0);
bool is_greater_than = (a > 1.0);
\end{lstlisting}

\subsubsection*{Methods}

As it happens for the \lstinline"Vector" class, the \lstinline"Matrix" class implements get and set methods for all the members. 

The \lstinline"get_rows_number" and \lstinline"get_columns_number" methods are very useful, 

\begin{lstlisting}
Matrix<MyClass> m(4, 2);
int rows_number = m.get_rows_number();
int columns_number = m.get_columns_number();
\end{lstlisting}

In order to set a new number of rows or columns to a \lstinline"Matrix" object, the \lstinline"set_rows_number" or \lstinline"set_columns_number" methods are used,

\begin{lstlisting}
Matrix<bool> m(1, 1);
m.set_rows_number(2);
m.set_columns_number(3);
\end{lstlisting}

% Initialization methods

A \lstinline"Matrix" can be initialized with a given value, at random with an uniform distribution or at random with a normal distribution, 

\begin{lstlisting}
Matrix<double> m(4, 2);
m.initialize(0.0);
m.initialize_uniform(-0.2, 0.4);
m.initialize_normal(-1.0, 0.25);
\end{lstlisting}

% Mathematical methods

A set of mathematical methods are also implemented for convenience. For instance, the \lstinline"dot" method computes the dot product of this \lstinline"Matrix" with a \lstinline"Vector" or with another \lstinline"Matrix",

\begin{lstlisting}
Matrix<double> m(4, 2, 1.0);
Vector<double> v(4, 2.0);
Vector<double> dot_product = m.dot(v);
\end{lstlisting}

% Utility methods

Finally, string serializing, printing, saving or loading utility methods are also implemented. For example, the use of the \lstinline"print" method is

\begin{lstlisting}
Matrix<bool> m(1, 3, false);
m.print();
\end{lstlisting}
