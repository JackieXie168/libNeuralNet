Inverse problems can be described as being opposed to direct
problems. In a direct problem the cause is given, and the effect is
determined. In an inverse problem the effect is given, and the cause
is estimated \cite{Kirsch1996}. There are two main types of inverse
problems: input estimation, in which the system properties and
output are known and the input is to be estimated; and properties
estimation, in which the the system input and output are known and
the properties are to be estimated \cite{Kirsch1996}. Inverse
problems are found in many areas of science and engineering.

%Inverse problems are those where a set of measured results is
%analyzed in order to get as much information as possible on a model
%which is proposed to represent a system in the real world. Exact
%inverse problems are related to most parts of mathematics. Applied
%inverse problems are the keys to other sciences. Hence the field,
%which is very wealthy, yields the best example of interdisciplinary
%research but it has nevertheless a strong individuality. The
%obtained results and explored directions of the 20th century are
%sketched in this review, with attempts to predict their evolution.
%
%Inverse problems are the problems that consist of finding an
%unknown property of an object, or a medium, from the observation
%of a response of this object, or medium, to a probing signal.
%Thus, the theory of inverse problems yields a theoretical basis
%for remote sensing and non-destructive evaluation
%\cite{Ramm2005}.

%For example, if an acoustic plane wave is scattered by an
%obstacle, and one observes the scattered field far from the
%obstacle, or in some exterior region, then the inverse problem is
%to find the shape and material properties of the obstacle. Such
%problems are important in identification of flying objects
%(airplanes missiles, etc.), objects immersed in water (submarines,
%paces of fish, etc.), and in many other situations. In geophysics
%one sends an acoustic wave from the surface of the earth and
%collects the scattered field on the surface for various positions
%of the source of the field for a fixed frequency, or for several
%frequencies. The inverse problem is to find the subsurface
%inhomogeneities. In technology one measures the eigenfrequencies
%of a piece of a material, and the inverse problem is to find a
%defect in this material, for example, a hole in a metal. In
%geophysics the inhomogeneity can be an oil deposit, a cave, a
%mine. In medicine it may be a tumor, or some abnormality in a
%human body. If one is able to find inhomogeneities in a medium by
%processing the scattered field on the surface, then one does not
%have to drill a hole in a medium. This, in turn, avoids expensive
%and destructive evaluation. The practical advantages of remote
%sensing are what makes the inverse problems important
%\cite{Ramm2005}.

An inverse problem is specified by the following concepts:

\begin{itemize}
\item[-] Mathematical model.
\item[-] Data set.
\item[-] Neural network.
\item[-] Performance functional.
\item[-] Training strategy.
\end{itemize}

\subsubsection*{Mathematical model}
\index{mathematical model}
\index{unknown variable}
\index{state variable}
\index{algebraic operator}
\index{differential operator}
\index{forcing term}

The mathematical model can be defined as a representation of the
essential aspects of some system which presents knowledge of that
system in usable form.

%Let us represent $\mathbf{y}(\mathbf{x})$ the vector of unknowns
%(inputs or properties) and
%$\mathbf{u}(\mathbf{x})$ the vector of state variables. The mathematical model, or state equation, relating
%unknown and state variables takes the form

%\begin{eqnarray}\label{InputEstimationEquation}
%\mathcal{L}(\mathbf{y}(\mathbf{x}),
%\mathbf{u}(\mathbf{x}),\mathbf{x}) = \mathbf{f},
%\end{eqnarray}

%\noindent where $\mathcal{L}$ is some algebraic or differential
%operator and $\mathbf{f}$ is some forcing term.


%The variables of an inverse problem, i.e., variables which describe
%the characteristics of the device for which the problem is
%specified, are interrelated according to the following operator
%equation \cite{Korovkin2007}:

%Equation (\ref{InverseProblemsEquation}) provides a means to
%calculate the variables $\mathbf{w}$ of the inverse problem for
%any $\mathbf{p}$.

\subsubsection*{Data set}
\index{observed data}

Inverse problems are those where a set of measured results is
analyzed in order to get as much information as possible on a
mathematical model which is proposed to represent a real system.

Therefore, a data set on the state variables is
needed in order to estimate the unknown variables of that system.

In general, that data set is invariably affected by noise and
uncertainty, which will translate into uncertainties in the system
inputs or properties.

\subsubsection*{Neural network}
\index{unknowns constraint}

The neural network represents here the inputs to the system or the properties of that system. 
They might include boundary conditions or bounds. 

\subsubsection*{Performance functional}
\index{state constraint}
\index{input constraint}\index{property constraint}
\index{boundary condition}\index{lower and upper bounds}
\index{admissible unknown}
\index{admissible state}

For inverse problems, the presence of restrictions is typical. 
State constraints are those conditions that the system needs to
hold. This type of restrictions depend on the particular problem.

In this way, an unknown which satisfies all the input and state
constraints is called an admissible unknown.


Also, a state which satisfies the state constraints is called an
admissible state.

The inverse problem provides a link between the outputs from the
model and the observed data. When formulating and solving inverse
problems the concept of error functional is used to specify the
proximity of the state variable to the
observed data.


Some common performance functionals for invers problems are the inverse sum squared
error or the inverse Minkowski error. Regularization
theory can also be applied here
\cite{Bucur2005}.

The solution of inverse problems is then reduced to finding the
extremum of a functional.

On the other hand, inverse problems might be ill-posed
\cite{Tikhonov1977}. A problem is said to be
well possed if the following conditions are true: (a) the solution to the problem exists;
(b) the solution is unique; and (c) the solution
is stable. This implies that for the
above-considered problems, these conditions can be violated.
Therefore, their solution requires application of special methods.
In this regard, the use of regularization theory is widely used
\cite{Engl2000}.

In some elementary cases, it is possible to establish analytic
connections between the sought inputs or properties and the observed
data. But for the majority of cases the search of extrema for the
error functional must be carried out numerically, and the so-called
direct methods can be applied.

\subsubsection*{Training strategy}

The training strategy is entrusted to solve the reduced function optimization problems. 
When possible, a quasi-Newton problem should be used. 
If the gradient of the performance function cannot be computed accurately, an evolutionary algorithm could be used instead. 

